% Created 2020-12-11 Fri 21:44
% Intended LaTeX compiler: pdflatex
\documentclass[11pt]{article}
\usepackage[utf8]{inputenc}
\usepackage[T1]{fontenc}
\usepackage{graphicx}
\usepackage{grffile}
\usepackage{longtable}
\usepackage{wrapfig}
\usepackage{rotating}
\usepackage[normalem]{ulem}
\usepackage{amsmath}
\usepackage{textcomp}
\usepackage{amssymb}
\usepackage{capt-of}
\usepackage{hyperref}
\date{\today}
\title{TD SIMD}
\hypersetup{
 pdfauthor={},
 pdftitle={TD SIMD},
 pdfkeywords={},
 pdfsubject={},
 pdfcreator={Emacs 27.1 (Org mode 9.3)}, 
 pdflang={English}}
\begin{document}

\maketitle
\tableofcontents

\uline{Directives}:

\begin{itemize}
\item Ce TD sera a rendre pour au plus tard Jeudi 17/12/2020 a 00:00 (minuit) aux adresses suivantes (selon qui est en charge de votre groupe): 

\begin{itemize}
\item \emph{mohammed-salah.ibnamar@uvsq.fr}
\item \emph{kevin.camus@uvsq.fr}
\end{itemize}

\item Il faudra utiliser comme sujet du courriel le format suivant (a respecter a la lettre): \textbf{\texttt{[ AoB TD\_simd - NOM PRENOM ]}}
\end{itemize}

\uline{Notes}:

\begin{itemize}
\item Vous pouvez effectuer votre exploration sur une VM car l'execution du code n'est pas importante.
\end{itemize}

\section{Introduction}
\label{sec:orga7d5a6a}

L'objectif de ce TD est d'explorer le jeu d'instructions SIMD present 
sur les processeurs x86 (Intel et AMD) dont vous disposez.

L'architecture x86 propose des instructions vectorielles/SIMD sous la forme d'extensions:

\begin{itemize}
\item \uline{\textbf{SSE}}: \textbf{Streaming SIMD Extensions} qui opere sur des vecteurs nommes \textbf{xmm?} (les registres disponibles vont \textbf{xmm0} a \textbf{xmm7}) qui ont une taille de \textbf{128bits} (16 octets)
\item \uline{\textbf{AVX}}: \textbf{Advanced Vector eXtensions} qui opere sur des vecteurs \textbf{ymm?} (\textbf{ymm0} a \textbf{ymm15}) qui ont une taille de \textbf{256bits} (32 octets) (SandyBridge)
\item \uline{\textbf{AVX2}}: \textbf{Advanced Vector eXtensions} qui opere sur des vecteurs \textbf{ymm?} (\textbf{ymm0} a \textbf{ymm15}) qui ont une taille de \textbf{256bits} (32 octets) ()
\item \uline{\textbf{AVX512}}: Similaire a AVX mais avec des registres \textbf{zmm?} (\textbf{zmm0} a \textbf{zmm31}) qui ont une taille de \textbf{512bits} (64 octets) - ce jeu d'instruction n'est disponible que sur les versions serveur d'Intel
\end{itemize}

\section{Travail a effectuer}
\label{sec:org2b20f88}
\subsection{Code}
\label{sec:orgedcb4b7}

Vous devez implementer plusieurs versions du code \textbf{dotprod} ci-dessous en deroulant autant de fois 
que vous desirez le corps de la boucle interne et comparer les codes assembleurs produits par le compilateur 
pour les differentes versions avec differentes options de compilation destinees a \textbf{produire du code assembleur vectoriel}.

\uline{Exemple du code non deroule}:

\begin{verbatim}

double dotprod(double *restrict a, double *restrict b, unsigned long long n)
{
    double d = 0.0;

    for (unsigned long long i = 0; i < n; i++)
       d += a[i] * b[i];

    return d;
}

\end{verbatim}

\uline{Exemple du code deroule 2 fois}:

\begin{verbatim}

double dotprod_unroll2(double *restrict a, double *restrict b, unsigned long long n)
{
    double d1 = 0.0;
    double d2 = 0.0;

    for (unsigned long long i = 0; i < n; i += 2)
       {
	  d1 += (a[i]     * b[i]);
	  d2 += (a[i + 1] * b[i + 1]);
       }

    return (d1 + d2);
}

\end{verbatim}

\subsection{Options de compilation}
\label{sec:org453610e}

\uline{Vous pouvez utiliser un ou plusieurs compilateurs, par exemple}: 

\begin{itemize}
\item GNU C Compiler \textbf{gcc}
\item Intel C Compiler \textbf{icc}
\item LLVM \textbf{clang}
\item AMD Optimizing C Compiler \textbf{aocc}.
\end{itemize}

\uline{Il faudra compiler le programme en suivant les etapes suivantes (pour \textbf{gcc} et \textbf{clang})}:

\begin{itemize}
\item \uline{Version de base}: -O1
\item \uline{Version legerement optimisee}: -O2
\item \uline{Version fortement optimisee 1}: -O3
\item \uline{Version fortement optimisee 2}: -Ofast
\item \uline{Version kamikaze}: -march=native -mtune=native -Ofast -funroll-loops -finline-functions -ftree-vectorize
\end{itemize}

Vous pouvez utiliser d'autres options de compilation: \textbf{RTFM}.

Pour generer du code assembleur utilisez l'option: \textbf{-S} avec tous les compilateurs

\section{Rendu}
\label{sec:orgc8d855a}

Vous devez fournir une archive contenant les codes sources et les codes assembleurs analyses en plus d'un \textbf{README} qui fournit
des explications pour chacune des versions de la fonction et si les instructions SIMD x86 (jeux d'instructions SSE et AVX) ont ete 
utilisees de facon scalaire (i.e. add[sd] ou add[ss]) ou vectorielle (i.e. add[pd] ou add[ps]).

\begin{itemize}
\item\relax [ss | sd]: \textbf{ss} pour \textbf{scalar} \textbf{simple} precision et \textbf{sd} pour \textbf{scalar} \textbf{double} precision (operations scalaires)
\item\relax [ps | pd]: \textbf{ps} pour \textbf{packed} \textbf{simple} precision et \textbf{pd} pour \textbf{packed} \textbf{double} precision (operations vectorielles)
\end{itemize}

Il est preferable de creer un \textbf{makefile} avec une regle par version (voir l'exemple fournit).

Il faudra aussi fournir un fichier contenant la version du compilateur utilise:

\begin{verbatim}

#Pour GCC
$ gcc --version > gcc_ver.txt

#Pour clang
$ clang --version > clang_ver.txt

\end{verbatim}

Et un fichier contenant les informations sur votre processeur:

\begin{verbatim}

$ cat /proc/cpuinfo > cpuinfo.txt

\end{verbatim}

Si vous decidez de faire tout le TD sur Compiler Explorer [1], vous n'avez pas besoin de fournir de \textbf{makefile}. Mais il faudra fournir les versions
des compilateurs utilises et les micro-architectures cibles [4] (-march=ARCH) et utiliser au minimum les trois compilateurs suivants: gcc, clang et icc.

\section{References:}
\label{sec:org2b954bc}

\begin{itemize}
\item 0. \uline{RTFM} (Read The Fucking Manual): \textbf{man gcc} ou \textbf{man clang} sur la console.

\item 1. \uline{Compiler explorer}: \url{https://gcc.godbolt.org/}
\item 2. \uline{Intel Intrinsics Guide:} \url{https://software.intel.com/sites/landingpage/IntrinsicsGuide/}
\item 3. \uline{GCC optimization options:} \url{https://gcc.gnu.org/onlinedocs/gcc/Optimize-Options.html}
\item 4. \uline{GCC x86 architectures:} \url{https://gcc.gnu.org/onlinedocs/gcc/x86-Options.html}
\item 5. \uline{Felix Cloutier's x86 instructions reference guide:} \url{https://www.felixcloutier.com/x86/}
\item 6. \uline{Agner Fog's optimization manuals:} \url{https://www.agner.org/optimize/}
\end{itemize}
\end{document}
